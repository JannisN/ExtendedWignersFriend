\documentclass[a4paper]{article}
\usepackage{fancyhdr}
\usepackage{physics}
\usepackage{graphicx}
\usepackage{hyperref}
\usepackage{csquotes}

\usepackage{subcaption}
\usepackage{floatrow}
\usepackage{fancyhdr}
\usepackage{afterpage}

\fancyhf{}
\pagestyle{fancy}
\rhead{\textcolor{gray}{Mock Draft}}

\title{Meine Antwort zum erweiterten Wigner's Freund Gedankenexperiment}
\author{Jannis Naske}

\pagestyle{fancy}
\fancyhf{}
\rhead{Jannis Naske}

\begin{document}
\maketitle
\afterpage{\cfoot{\thepage}}

\section*{Abstract}
In diesem Dokument schlage ich ein neues Postulat für die Quantenmechanik vor, das den Widerspruch im Artikel zum erweiterten Wigner's Freund Gedankenexperiment von Renner und Frauchiger auflöst.

\section*{Das Postulat}
Das Postulat lautet wie folgt:
\begin{itemize}
	\item[] Die Quantenmechanik mit ihren Messungen muss immer aus der Many-Worlds Perspektive beschrieben werden.
\end{itemize}
Der Grund wieso dieses Postulat früher nicht gebraucht wurde, ist, dass dieses Gedankenexperiment eine Änderung im Bezug auf andere (Gedanken)Experimente enthält:
Zwei Agenten, wobei beide Agenten den anderen jeweils in einer Superposition sehen, machen Aussagen übereinander.
Dieses Postulat erzeugt praktisch keine Änderungen bei älteren Experimenten, da sie diese Eigenschaft nicht besitzen.
Jedoch hat es Auswirkungen auf folgende Aussage:
\begin{itemize}
	\item[] \textbf{Statement $\overline{F}^{n:02}$:} ``I am certain that $W$ will observe $w$ = fail at time $n$:31.''
\end{itemize}
Das Problem ist, dass er diese Aussage nur macht, falls er $r$ = tails gemessen hat. Das Postulat besagt nun, dass er bei der Herleitung seines Statements sich selbst
auch aus der Many-Worlds Perspektive beschreiben muss. Das heisst, obwohl er aus seiner Sicht entweder $\ket{h}$ oder $\ket{t}$ gemessen hat,
muss er trotzdem beide Möglichkeiten miteinbeziehen. Wir starten die Analye ab $n$:11:
\begin{align*}
\frac{1}{\sqrt{3}}\ket{h}\ket{\downarrow} + \sqrt{\frac{2}{3}}\ket{t}\qty(\frac{1}{\sqrt{2}} \ket{\downarrow} + \frac{1}{\sqrt{2}} \ket{\uparrow})
\end{align*}
Wir stellen nun diesen Zustand in den Basen dar, in denen $\overline{W}$ und $W$ messen:
$\overline{W}$ misst in der Basis $\qty{\ket{+}_{r}, \ket{-}_{r}}$,
mit $\ket{+}_{r} = \frac{1}{\sqrt{2}} \ket{h} + \frac{1}{\sqrt{2}} \ket{t}$, $\ket{-}_{r} = \frac{1}{\sqrt{2}} \ket{h} - \frac{1}{\sqrt{2}} \ket{t}$,
und $W$ in der Basis $\qty{\ket{+}_{w}, \ket{-}_{w}}$,
mit $\ket{+}_{w} = \frac{1}{\sqrt{2}} \ket{\downarrow} + \frac{1}{\sqrt{2}} \ket{\uparrow}$, $\ket{-}_{w} = \frac{1}{\sqrt{2}} \ket{\downarrow} - \frac{1}{\sqrt{2}} \ket{\uparrow}$:
\begin{align*}
\frac{3}{\sqrt{12}}\ket{+}_{r}\ket{+}_{w} + \frac{1}{\sqrt{12}}\ket{+}_{r}\ket{-}_{w} - \frac{1}{\sqrt{12}}\ket{-}_{r}\ket{+}_{w} + \frac{1}{\sqrt{12}}\ket{-}_{r}\ket{-}_{w}
\end{align*}
Man sieht nun deutlich, dass $W$ auch $w$ = ok messen kann aus $\overline{F}$'s Sicht.
Unter der Annahme, dass die Annahmen Q, C und S korrekt sind, wird dieses Postulat sozusagen erzwungen.
Um dieses Postulat zu überprüfen kann man das Gedankenexperiment in wirklichkeit ausführen.

\section*{Referenz}
\begin{itemize}
	\item \url{https://www.ncbi.nlm.nih.gov/pmc/articles/PMC6143649/}
\end{itemize}

\end{document}