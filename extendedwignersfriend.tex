\documentclass[a4paper]{article}
\usepackage{fancyhdr}
\usepackage{physics}
\title{Meine Antwort zum erweiterten Wigner's Freund Gedankenexperiment}
\author{Jannis Naske}

\pagestyle{fancy}
\fancyhf{}
\rhead{Jannis Naske}

\begin{document}
\pagenumbering{gobble}
\maketitle
\pagenumbering{arabic}

\section*{Abstract}
In diesem Dokument schlage ich zwei mögliche Korrekturen zum erweiterten Wigner's Freund Gedankenexperiment von Renner und Frauchiger vor.
Durch diese Verbesserungen wird der Widerspruch vernichtet, und alle drei Annahmen, \textbf{(Q)}, \textbf{(C)} und \textbf{(S)}, bleiben unverletzt.

\section*{Der erste Fehler}
Im Artikel von Renner und Frauchiger wird folgendes Statement hergeleitet:
\begin{itemize}
	\item \textbf{Statement 1 by} $F_1$: ``If I get $t$, I know that $W_2$ will measure $plus$''
\end{itemize}
Der Beweis, welcher benutzt wird, ist folgender(ich lasse in diesem Dokument die doppelten Symbole weg, da dies in diesem Fall redundante Information ist):\\\\
Nachdem $F_1$ $t$ gemessen hat, setzt er den Spin für $F_2$ in die Superposition $\frac{1}{\sqrt{2}} \ket{\downarrow} + \frac{1}{\sqrt{2}} \ket{\uparrow}$.
In der Basis $\qty{\ket{+}_{L_2}, \ket{-}_{L_2}}$, mit $\ket{+}_{L_2} = \frac{1}{\sqrt{2}} \ket{\downarrow} + \frac{1}{\sqrt{2}} \ket{\uparrow}$, $\ket{-}_{L_2} = \frac{1}{\sqrt{2}} \ket{\downarrow} - \frac{1}{\sqrt{2}} \ket{\uparrow}$,
ist diese Superposition dargestellt als $\ket{+}_{L_2}$, und $W_2$ wird somit $\ket{+}_{L_2}$ messen, und die Aussage folgt.\\\\
Jedoch wurde bei diesem Beweis weggelassen, dass die Superposition durch das Messen von $W_1$ verändert wird.
Wenn $W_1$ nach Annahme $\ket{-}_{L_1} = \frac{1}{\sqrt{2}}\ket{h} + \frac{1}{\sqrt{2}}\ket{t}$ misst, geht die Superposition, nach dem Artikel,
in $\ket{-}_{L_1}\ket{\uparrow} = \frac{1}{\sqrt{2}}\ket{h}\ket{\uparrow} - \frac{1}{\sqrt{2}}\ket{t}\ket{\uparrow} = \qty(\frac{1}{\sqrt{2}} \ket{h} - \frac{1}{\sqrt{2}} \ket{t})\qty(\ket{+}_{L_2} - \ket{-}_{L_2}) = \frac{1}{2}\ket{h}\ket{+}_{L_2} - \frac{1}{2}\ket{t}\ket{+}_{L_2} - \frac{1}{2}\ket{h}\ket{-}_{L_2} + \frac{1}{2}\ket{t}\ket{-}_{L_2}$ über. Es ist also doch möglich, dass $W_2$ $\ket{t}\ket{-}_{L_2}$ misst, und Statement 1 stellt sich als falsch heraus.\\\\
Zum Schluss misst $W_2$ nach Annahme noch $\ket{-}_{L_2}$, und der Zustand geht in $\frac{1}{\sqrt{2}}\ket{t}\ket{-} - \frac{1}{\sqrt{2}}\ket{h}\ket{-} = \frac{1}{2}\ket{t}\ket{\downarrow} - \frac{1}{2}\ket{t}\ket{\uparrow} - \frac{1}{2}\ket{h}\ket{\downarrow} + \frac{1}{2}\ket{h}\ket{\uparrow}$ über.

\end{document}