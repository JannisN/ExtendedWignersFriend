\documentclass[a4paper]{article}
\usepackage{fancyhdr}
\usepackage{physics}
\usepackage{graphicx}
\usepackage{hyperref}
\usepackage{csquotes}

\usepackage{subcaption}
\usepackage{floatrow}
\usepackage{fancyhdr}
\usepackage{afterpage}

\fancyhf{}
\pagestyle{fancy}
\rhead{\textcolor{gray}{Mock Draft}}

\title{My Answer To The Extended Wigner's Friend Thaught Experiment}
\author{Jannis Naske}

\pagestyle{fancy}
\fancyhf{}
\rhead{Jannis Naske}

\begin{document}
\maketitle
\afterpage{\cfoot{\thepage}}

\section*{Abstract}
In this document I propose a new postulate in quantum mechanics, that resolves the contradiction in the paper ``Quantum theory cannot consistently describe the use of itself'' by Renner and Frauchiger.

\section*{The Postulate}
The postulate looks like this:
\begin{itemize}
	\item[] Quantum mechanics and its measurements have to be described from the view of the many worlds perspective.
\end{itemize}
The reason why this postulate wasn't needed earlier in the past, is that this thaught experiment has a different property compared to other (thaught)experiments:
Two agents, who see each other in a superposition, make statements about each other.
This postulate practically has no impact on other, older experiments, since those don't have this property.
However, it influences the following statement:
\begin{itemize}
	\item[] \textbf{Statement $\overline{F}^{n:02}$:} ``I am certain that $W$ will observe $w$ = fail at time $n$:31.''
\end{itemize}
The problem is that $\overline{F}$ only makes this statement when he measured $r$ = tails. The postulate now says that in his derivation of the statement,
he now has to describe himself from the view of the many worlds perspectve. This means that no matter if he measured $\ket{h}$ or $\ket{t}$,
he still has to consider the situation where he would have measured the other one. We start the analysis from $n$:11 on:
\begin{align*}
\frac{1}{\sqrt{3}}\ket{h}\ket{\downarrow} + \sqrt{\frac{2}{3}}\ket{t}\qty(\frac{1}{\sqrt{2}} \ket{\downarrow} + \frac{1}{\sqrt{2}} \ket{\uparrow})
\end{align*}
Now we describe this state in the bases that $\overline{W}$ and $W$ use in their measurements:
$\overline{W}$ measures in the basis $\qty{\ket{+}_{r}, \ket{-}_{r}}$,
with $\ket{+}_{r} = \frac{1}{\sqrt{2}} \ket{h} + \frac{1}{\sqrt{2}} \ket{t}$, $\ket{-}_{r} = \frac{1}{\sqrt{2}} \ket{h} - \frac{1}{\sqrt{2}} \ket{t}$,
and $W$ in the basis $\qty{\ket{+}_{w}, \ket{-}_{w}}$,
where $\ket{+}_{w} = \frac{1}{\sqrt{2}} \ket{\downarrow} + \frac{1}{\sqrt{2}} \ket{\uparrow}$, $\ket{-}_{w} = \frac{1}{\sqrt{2}} \ket{\downarrow} - \frac{1}{\sqrt{2}} \ket{\uparrow}$:
\begin{align*}
\frac{3}{\sqrt{12}}\ket{+}_{r}\ket{+}_{w} + \frac{1}{\sqrt{12}}\ket{+}_{r}\ket{-}_{w} - \frac{1}{\sqrt{12}}\ket{-}_{r}\ket{+}_{w} + \frac{1}{\sqrt{12}}\ket{-}_{r}\ket{-}_{w}
\end{align*}
As one can see, $W$ can measure $w$ = ok, even from $\overline{F}$'s perspective.
Under the assumption, that the assumptions Q, C and S are correct, this postulate is forced.
To actually test this postulate, one can execute this thaught experiment in reality and compare the result to the prediction of this postulate.

\section*{Reference}
\begin{itemize}
	\item \url{https://www.ncbi.nlm.nih.gov/pmc/articles/PMC6143649/}
\end{itemize}

\end{document}